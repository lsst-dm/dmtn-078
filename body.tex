\section{Introduction}
In this brief note we wish to discuss some potential routes to collaboration with Google. We will revisit some of the potential ways forward to the cloud from \citeds{DMTN-072} and try to quantify what a Proof of Concept might look like. There is also scope for technical collaboration on infrastructure.

If we go ahead with one specific POC we should define more clearly the goal and duration of it and how that might feed into future operations.

{\color{red} Margaret/Michelle - should we have - expected results/metrics could also be the next round and can you look at th technical collaboration section. }
\section{Technical collaboration}

LSST is already a heavy user of Kubernetes (K8s) both internally and via Google Cloud.
We have experienced specific technical issues running on our own K8s enabled hardware such as excessive start-up times\footnote{Solved for now.} and security concerns around integration with the GPFS filesystem.
We believe we are pushing the boundaries of deployments of this sort: any help would be appreciated, and we provide a testbed outside Google for this type of work.

\section{Cloud proofs of concept}\label{sect:pocs}

Much of the LSST Data Management System (DMS) – and, certainly, everything described below — is deployable using Kubernetes.
This provides us with a lot of flexibility to port our system across service offerings, and would enable us to easily adopt a hybrid cloud plus on-premises infrastructure.

Moving to a cloud-based infrastructure could potentially save on personnel, as no hands-on hardware maintenance would be required.
Although this is equivalent to a relatively small fraction of the construction budget, it would represent a substantial sum dedicated to non-core-business during operations.

We can probably not move wholescale to the cloud: we are committed to providing a physical Data Access Center (DAC) in Chile, and some physical hardware must remain on the mountain and in the Commissioning Cluster.
However, there are potentially a number of opportunities to migrate a subset of DM services to the cloud if we could see a sensible way forward.

\subsection{Qserv: the LSST database}\label{sect:qserv}

The bespoke Qserv database system \citedsp{LDM-135}, under development at SLAC, has not yet been tested in a cloud based environment.
However it is now deployable with Kubernetes, and no longer requires special hardware: physically attached storage is needed, but this is available on cloud offerings.

Qserv's performance characteristics are well understood, and would form an excellent testbed for LSST operations in a cloud environment.
However, Qserv itself is primarily useful in the context of the Science Platform (\secref{sect:platform}), so the longer term benefits of a Qserv-only deployment would be limited.

\subsubsection{Potential needs} \label{sect:qservneeds}

To set up a Qserv instance we would need at least 40 large nodes with physically attached storage on the order of 5 terabytes per node.
To run a set of convincing tests we would need that up for order two months.

This would be a  demonstration only - the final catalogs (2032) will be order 15PB and the number of nodes and attached storage eventually has to scale to that size.

\subsection{Cloud-based Science Platform} \label{sect:platform}

The LSST Science Platform \citedsp{LSE-319} envisons three key ways in which scientists will interact with LSST data: through a visualization portal, a Jupyter notebook-based interface and through a variety of web services.
These serve as an interface to images stored on disk and catalog data stored in the Qserv system (\secref{sect:qserv}).
This is an intrinsically cloud-oriented approach to the problem of accessing large volumes of LSST data: it is based upon user code being co-located with the data on which it is running.

A key benefit of a cloud-based Science Platform would be scalability: when user demands exceeds the 10\% of the compute budget dedicated to serving them, more capacity would at least be available even if it had to be purchased on demand.
There is no analogue to this in terms of on-premises infrastructure as cloud bursting from our internal cloud infrastructure to a commercial provider would require transferring potentially large amounts of data.

\subsubsection{Aside: Public Data Releases}

LSST data becomes public two years after its initial release.
However there is no project budget allocated to serve this public data, although it remains scientifically valuable.
One imagines that a public, cloud-based version of the Science Platform serving old data could be a valuable resource e.g. for enabling science in underdeveloped countries.

one could envision a public version of the Science Platform serving the old data as something potentially interesting for some foundations/companies e.g. to enable science in underdeveloped countries.

\subsubsection{Potential needs}

All the Science Platform components are deployable with Kubernetes.
The Qserv database component is a fixed size resource as discussed in \secref{sect:qservneeds}.
In addition one or preferably two servers should be provisioned for the web services.

Alongside that one needs to have the JupyterHub environment \footnote{see https://github.com/lsst-sqre/jupyterlabdemo}; depending on the assumed load, this is relatively modest as it requires only $\sim2$ servers to set up, and it is recommended to have 2 CPUs per simultaneous user.
For a proof of concept let's assume we would go with 20 simultaneous users to 40 CPUs or 10 nodes depending on the type of node.
Each user should also have around 4GB of RAM.
Ideally, we would also have a batch system to control the compute resources but perhaps for a proof of concept this may be treated as a desire rather than a requirement.

Firefly also requires at least a pair of  server - theses should be 32 cores
with 128 GB memory. In addition these should have a shared disk volume order 500GB, preferably SSD.

Finally there is a filesystem to store the image data.
Our current code assumes a POSIX filesystem, but we have made some modifications towards supporting an object-store back end.
Additional investment in this direction is unlikely to happen before Fall 2018.
LSST will produce over its lifetime around 60\,PB of raw image data, the final data volume including the processed images is estimated to be around 0.5\,EB.
For the POC 1\,PB would be sufficient to evaluate performance and management of a large disk volume.

For the proof of concept we could leave out the Prompt Products Database (this is a conventional e.g. Oracle database).

\subsection{Cloud based prompt processing}\label{sect:pp}

\textit{Prompt processing} \citedsp{LDM-151} is the umbrella term used to describe processing which produces data products continuously while LSST operates.
Broadly, this falls into three categories:

\begin{enumerate}

\item{\textit{Image reduction and differencing}, in which images are received from the camera, calibrated, and compared to deep template images of the same part of sky to identify transient and variable sources;}
\item{\textit{Alert distribution}, in which notifications of transients and variables are distributed to the community;}
\item{\textit{Moving objects processing}, in which solar system objects are identified and their orbits tracked.}

\end{enumerate}

We expect to issue around $10^7$ alerts per night during normal operations.
Further, the project is required to make these alerts available to the community within no more than 60\,s of the telescope shutter closing.
This imposes stringent latency and throughput requirements on items 1 \& 2 above.
Moving object processing can be run during the day, and is therefore a less challenging — and for this purpose less interesting — use case.

One night of LSST observing generates appoximately 20\,TB of data in a ``bursty'' fashion (we visit each field for a total of 37\,s, taking two consecutive exposures which are combined by the data processing system).
In order to meet our latency requirements, we have invested in fast networking to enable rapid transfer of this data to processing systems at NCSA.

Once on the compute systems, data from each of the 189 CCDs in the camera is processed in parallel: broadly, we expect a single CPU with access to 4\,GB RAM to handle each CCD independently, taking somewhat less than a minute to complete.
In operations, we anticipate deploying two separate clusters, or ``chains'', with each processing alternating visits.

These processing chains will then feed the results of their processing to the alert distirbution system, based on Apache Kafka\footnote{\url{http://kafka.apache.org}}, for distribution to the community.
A prototype of this system has already been deployed on Amazon AWS \citeds{DMTN-028}.

This suggests two, related, proof of concept exercises:

\begin{itemize}

\item{Demonstrate rapid ingest and image processing;}
\item{Demonstrate at-scale alert distribution.}

\end{itemize}

\subsubsection{Potential needs}

Demonstrating image processing at scale could be achieved with a single processing chain (ie, 189 CPU cores, with access to 4\,GB RAM per core).
Around 60\,TB disk storage is required for a single night of data (including processed data products).
However, a smaller dataset could be defined for testing purposes.

Prompt image processing would also require a database instance to serve as the Prompt Products Database (PPDB).

Deploying a realistic alert distribution system would require three systems, each with access to 24 CPU cores and 80\,GB RAM.

\section{Conclusion}
A number of potential POCs are discussed above with approximates sizes/needs.
We should pick one or more to develop further.
